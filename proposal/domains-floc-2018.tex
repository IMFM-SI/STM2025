\documentclass{article}

\usepackage{times}
\usepackage[utf8]{inputenc}
\usepackage{color}
\definecolor{grey}{rgb}{0.4,0.4,0.4}
\newcommand{\grey}{\textcolor{grey}}

\usepackage[english]{babel}
\usepackage{url}
\usepackage{booktabs}

\title{Proposal for a FLoC workshop \\
  in honour of Dana Scott's 85th~birthday \\
  and 50~years of domain theory}

\author{Andrej Bauer \and Martín Escardó}

\date{}

\begin{document}

\maketitle

\begin{abstract}
  We propose a FLoC~2018 workshop on domain theory, affiliated with LICS, to celebrate Dana Scott's 85th birthday and 50~years of domain theory.
\end{abstract}

\section{Scientific justification}

Fifty years ago, Dana Scott introduced domain theory for the purposes
of denotational semantics of programming languages when he was in
Oxford, where he worked with Christopher Strachey. This work has had a
vast and lasting impact on logic, computer science, and
mathematics. We plan to celebrate 50 years of domain theory and Dana's
85th birthday in the \emph{Domains Workshop} series, which we propose
to be part of FLoC.

The applications of domain theory include programming logics (LCF),
design of programming languages, models of the lambda calculus,
applications to recursion theory (higher-type computability,
Kleene-Kreisel countable functionals), general topology (injective
spaces, function spaces, locally compact spaces, Stone duality),
topological algebra (Lawson semilattices) and analysis (measure,
integration, dynamical systems). Moreover, these applications are
related --- for example, Stone duality has given rise to a logic of
observable properties of computational processes.

The Domains workshop series is aimed at computer scientists and
mathematicians alike who share an interest in the mathematical
foundations of computation. The workshop series focuses on domains,
their applications in mathematics and computer science, and related
topics.  Previous meetings were held in Darmstadt (1994, 1999, 2004),
Braunschweig (1996), Munich (1997), Siegen (1998), Birmingham (2002),
Novosibirsk (2007), Brighton (2008), Swansea (2011), Paris (2014), and
Cork (2015).

\section{Organization}

\subsection{Workshop title}

\begin{center}
\large
\textbf{Domains Workshop}\\
\emph{50 years of domain theory}
\end{center}

\subsection{Workshop organizers}

The workshop will be organized by Andrej Bauer and Martín Escardó:
%
\begin{quote}
Andrej Bauer \\
Faculty of Mathematics and Physics \\
University of Ljubljana, Slovenia \\
\url{andrej.bauer@andrej.com} \\
\url{http://www.andrej.com/} \\
~ \\
Martín Escardó \\
School of Computer Science \\
University of Birmingham, UK \\
\url{m.escardo@cs.bham.ac.uk} \\
\url{http://www.cs.bham.ac.uk/~mhe/}
\end{quote}

\noindent %
They will act as an organizing committee, taking care of all organizational matters.
Through communication with the leading researchers in the area of domain theory, they will
help form a programme committee.

\subsection{Affiliated conference}

A workshop on domain theory is naturally affiliated with \emph{Logic in computer science
  (LICS),} as domain theory is one of the standard LICS topics.

As discussed above, we plan to make the meeting an instance of
the Domains Workshop series. The organizers of the series, Klaus
Keimel and Achim Jung, support the idea. 

\subsection{Anticipated number of participants}
\label{sec:antic-numb-part}

We estimate that the workshop would be attended by 70 to 80 participants. While this is a
bit higher than a typical Domains Workshop, we anticipate good attendance because of the
festive nature of the meeting, and synergies created by FLoC.

\section{Proposed format and agenda}

\subsection{Programme committee}

A program committee would likely consist of four senior researchers, with expertise in all
aspects of domain theory, and good knowledge (first-hand when possible) of the history of
domain theory.

% A compendium member (Mislove, Lawson)
% Programming language (Plotkin, Pitts)
% Synthetic (Hyland, Rosolini, Simpson)
% Higher-type Computability (Ulrich Berger, Dag Normann, John Longley)

\subsection{Workshop format}

The two-day meeting will open with a plenary session by Dana Scott.
The meeting will have three further plenary invited talks, and a
number of short invited and contributed talks covering current topics
in domain theory. A provisional organization of the talks is shown in
Table~\ref{tab:layout}. The ratio between invited and contributed
talks is to be determined by the programme committee. Depending on the
number and quality of submissions, the contributed talks may be a bit
shorter than the invited talks. In any case, we shall comply with the
standard FLoC workshop format.

\begin{table}[ht]
  \centering
\begin{tabular}{lll}
\multicolumn{3}{c}{Day 1} \\ \midrule
Morning   & Dana Scott's plenary talk & 90 minutes \\
          &                                & \grey{\emph{coffee}}\\
          & Two short talks                & 60 minutes \\
          &                                & \grey{\emph{lunch}} \\
Afternoon & Invited plenary talk           & 60 minutes \\
          & Short talk                     & 30 minutes \\
          &                                & \grey{\emph{coffee}} \\
          & Short talks                    & up to 120 minutes \\[2ex]
\multicolumn{3}{c}{Day 2} \\ \midrule
Morning   & Invited plenary talk           & 60 minutes \\
          & Short talk                     & 30 minutes \\
          &                                & \grey{\emph{lunch}}  \\
Morning   & Two short talks                & 60 minutes \\
          &                                & \grey{\emph{lunch}}  \\
Afternoon & Invited plenary talk           & 60 minutes \\
          & Short talk                     & 30 minutes \\
          &                                & \grey{\emph{coffee}} \\
Afternoon & Short talks                    & up to 120 minutes \\
\end{tabular}
  \caption{A possible workshop format}
  \label{tab:layout}
\end{table}

\subsection{Procedures for selecting papers and participants}

The program committee will select plenary speakers, invited speakers, as well as
contributed papers based on submitted two-page abstracts. The selection of plenary
speakers will reflect the breadth and historic development of domain theory, while the
short invited and contributed talks will be devoted to current topics in domain theory.

% Many other people are also qualified to represent the depth of domain theory, including:
% Abramsky, Berger, Erhard, Edalat, Hyland, Longley, Normann, Simpson.

\subsection{Plans for dissemination}

We plan to produce post-proceedings in a scientific journal to be determined by the
organizing committee. This is common practice for the Domains Workshop series, which published post-proceedings in journals such as \emph{Mathematical Structure in Computer Science} and \emph{Electronic Notes in Theoretical Computer science}. Other possible journals are \emph{Logical Methods in Computer Science} and \emph{Leibniz International Proceedings in Informatics}.

\subsection{Duration}

Give the breadth and depth of domain theory, as well as Dana Scott's work, and the many
connections of domain theory to logic and computer science, we request \textbf{2 days} for
this workshop. Our preference is for the workshop to take place after LICS and during~FLoC.

\end{document}
