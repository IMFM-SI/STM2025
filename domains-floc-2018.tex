\documentclass{article}

\usepackage{color}
\definecolor{grey}{rgb}{0.4,0.4,0.4}
\newcommand{\grey}{\textcolor{grey}}

\usepackage[english]{babel}
\usepackage{url}

\title{Workshop in honour of Dana Scott's 85th birthday and 50 years of domain theory}

\author{Andrej Bauer and Mart\'{\i}n Escard\'o}

\date{}

\begin{document}

\maketitle

\section{Scientific justification}

* On the occasion of Dana Scott 85th birthday and 50 years of
domain theory.

The Workshop on Domains is aimed at computer scientists and
mathematicians alike who share an interest in the mathematical
foundations of computation. The workshop will focus on domains,
their applications and related topics.  Previous meetings were
held in Darmstadt (94,99,04), Braunschweig (96), Munich (97),
Siegen (98), Birmingham (02), Novosibirsk (07) and Brighton (08),
Swansea (2011), Paris (2014), Cork (2015).

SCOPE

    Domain theory has had applications to programming language
semantics and logics (lambda-calculus, PCF, LCF), recursion theory
(Kleene-Kreisel countable functionals), general topology (injective
spaces, function spaces, locally compact spaces, Stone duality),
topological algebra (compact Hausdorff semilattices) and analysis
(measure, integration, dynamical systems). Moreover, these
applications are related - for example, Stone duality has given
rise to a logic of observable properties of computational
processes.

\section{Organization}

\subsection{Organizing committee}

\begin{quote}
Andrej Bauer \\
Faculty of Mathematics and Physics \\
University of Ljubljana, Slovenia \\
\url{andrej.bauer@andrej.com} \\
\url{http://www.andrej.com/} \\
~ \\
Mart\'{\i}n Escard\'o \\
School of Computer Science \\
University of Birmingham, UK \\
\url{m.escardo@cs.bham.ac.uk} \\
\url{http://www.cs.bham.ac.uk/~mhe/}
\end{quote}

\subsection{Potential programme committee}

[[Discussion]]

A compendium member (Mislove, Lawson)
Programming language (Plotkin, Pitts)
Synthetic (Hyland, Rosolini, Simpson)
Higher-type Computability (Ulrich Berger, Dag Normann, John Longley)

* organisational part:

*        contact information for the workshop organizers;

*        proposed affiliated conference: LICS


* estimate of the number of workshop participants: 55-65

\section{proposed format and agenda}

The main invited speaker will be Dana Scott, opening the meeting.
The plenary speaker should include former students representing Dana Scott's the wide interests (e.g. Fourman (topos theory, topology and constructive mathematics), Rosolini (sythetic domain theory), Birkedal (programming languages), and collaborators (e.g. Lawson (topology and topological algebra) among others). The precise selection will be made by the programme committee.

We will also have non-plenary invited speakers in various topics including higher-type computability, 

\begin{tabular}{|l|l|l|l|}
\hline & & duration & \\ \hline
\hline Day 1 & & & \\ \hline 
& Morning 1 & 90 & Dana Scott \\
& & & \grey{\emph{coffee}}   \\
& Morning 2 & 60 & 2 invited \\
& & & \grey{\emph{lunch}}   \\
& Afternoon 1 & 60 & 1 plenary \\
&  & 30 & 1 invited \\
& & & \grey{\emph{coffee}} \\
& Afternoon 2 & 100 & 4 contributed \\
\hline Day 2 & & & \\ \hline
& Morning 1 & 60 & 1 plenary \\
&           & 30 & 1 invited \\
& & & \grey{\emph{lunch}}  \\
& Morning 2 & 60 & 2 invited \\
& & & \grey{\emph{lunch}}  \\
& Afternoon 1 & 60 & 1 plenary \\
& & 30 & 1 invited \\
& & & \grey{\emph{coffee}}  \\
& Afternoon 2 & 100 & 4 contributed \\ \hline  
\end{tabular}

\subsection{Procedures for selecting papers and participants}

The program committee will select plenary speakers and invited speakers, as well as contributed papers based on submitted 2-page abstracts. 


Many other people are also qualified to represent the depth of domain theory, including: Abramsky, Berger, Erhard, Edalat, Hyland, Longley, Normann, Simpson. 

\subsection{Plans for dissemination}

We plan to produce post-proceedings in a journal to be determined by the organizing committee.

\subsection{Duration}

Give the breadth and depth of Dana Scott's work, we request 2 days for
this workshop.  Our preference is for the workshop to take place after
LICS.  during FLoC.


\end{document}
