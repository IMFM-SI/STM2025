\documentclass{article}

\usepackage{times}
\usepackage[utf8]{inputenc}
\usepackage{color}
\definecolor{grey}{rgb}{0.4,0.4,0.4}
\newcommand{\grey}{\textcolor{grey}}

\usepackage[english]{babel}
\usepackage{url}
\usepackage{booktabs}

\title{Proposal for a FLoC workshop\\
  in honour of  Dana Scott's 85th~birthday\\
  and 50~years of domain theory}

\author{Andrej Bauer \and Martín Escardó}

\date{}

\begin{document}

\maketitle

\begin{abstract}
  We propose a FLoC~2018 workshop on domain theory, affiliated with LICS. There will be a
  special session in honor of Dana Scott's 85th birthday and 50~years of domain theory.
\end{abstract}

\section{Scientific justification}

NOTE: state somewhere that denotational semantics was born in Oxford.

The Workshop on Domains is aimed at computer scientists and
mathematicians alike who share an interest in the mathematical
foundations of computation. The workshop will focus on domains,
their applications and related topics.  Previous meetings were
held in Darmstadt (94,99,04), Braunschweig (96), Munich (97),
Siegen (98), Birmingham (02), Novosibirsk (07) and Brighton (08),
Swansea (2011), Paris (2014), Cork (2015).

SCOPE

    Domain theory has had applications to programming language
semantics and logics (lambda-calculus, PCF, LCF), recursion theory
(Kleene-Kreisel countable functionals), general topology (injective
spaces, function spaces, locally compact spaces, Stone duality),
topological algebra (compact Hausdorff semilattices) and analysis
(measure, integration, dynamical systems). Moreover, these
applications are related --- for example, Stone duality has given
rise to a logic of observable properties of computational
processes.

\section{Organization}

\subsection{Workshop title}

\begin{center}
\large
\textbf{Domains Workshop}\\
\emph{50 years of domain theory}
\end{center}

\subsection{Workshop organizers}

The workshop will be organized by Andrej Bauer and Martín Escardó:
%
\begin{quote}
Andrej Bauer \\
Faculty of Mathematics and Physics \\
University of Ljubljana, Slovenia \\
\url{andrej.bauer@andrej.com} \\
\url{http://www.andrej.com/} \\
~ \\
Martín Escardó \\
School of Computer Science \\
University of Birmingham, UK \\
\url{m.escardo@cs.bham.ac.uk} \\
\url{http://www.cs.bham.ac.uk/~mhe/}
\end{quote}

\noindent %
They will act as an organizing committee, taking care of all organizational matters.
Through communication with the leading researchers in the area of domain theory, they will
help form a programme committee.

\subsection{Affiliated conference}

A workshop on domain theory is naturally affiliated with \emph{Logic in computer science
  (LICS),} as it is one of the standard LICS topics.

We also plan to make the meeting an instance of the Domains Workshop series. The
organizers of the series, Klaus Keimel and Achim Jung, support the idea. In this case the
event would be a FLoC workshop as well as the 13th Domains Workshop.

\subsection{Anticipated number of participants}
\label{sec:antic-numb-part}

We estimate that the workshop would be attended by 70 to 80 participants. While this is a
bit higher than a typical Domains Workshop, we anticipate good attendance because of the
festive nature of the meeting, and synergies created by FLoC.

\section{Proposed format and agenda}

\subsection{Programme committee}

A program committee would likely consist of four senior researchers, with expertise in all
aspects of domain theory, and good knowledge (first-hand when possible) of the history of
domain theory.

% A compendium member (Mislove, Lawson)
% Programming language (Plotkin, Pitts)
% Synthetic (Hyland, Rosolini, Simpson)
% Higher-type Computability (Ulrich Berger, Dag Normann, John Longley)

\subsection{Workshop format}

The two-day meeting will open with a special plenary session in honor of Dana Scott's
85th~birthday and a celebration of 50~years of domain theory. The meeting will have three
plenary invited talks, and number of short invited and contributed talks covering current
topics in domain theory. A provisional organization of the talks is shown in
Table~\ref{tab:layout}. The ratio between invited and contributed talks is to be
determined by the programme committee. Depending on the number and quality of submissions,
the contributed talks may be a bit shorter than the invited talks. In any case, we shall
comply with the standard FLoC workshop format.

% The main invited speaker will be Dana Scott, opening the meeting. The plenary speaker
% should include former students representing Dana Scott's the wide interests (e.g. Fourman
% (topos theory, topology and constructive mathematics), Rosolini (sythetic domain theory),
% Birkedal (programming languages), and collaborators (e.g. Lawson (topology and topological
% algebra) among others). The precise selection will be made by the programme committee.

% \begin{table}[ht]
%   \centering
% \begin{tabular}{lll}
% \multicolumn{3}{c}{Day 1} \\ \midrule
% Morning   & Session in honor of Dana Scott & 90 minutes \\
%           &                                & \grey{\emph{coffee}}\\
%           & Two invited talks              & 60 minutes \\
%           &                                & \grey{\emph{lunch}}   \\
% Afternoon & Invited plenary talk           & 60 minutes \\
%           & Invited talk                   & 30 minutes \\
%           &                                & \grey{\emph{coffee}} \\
%           & Four contributed talks         & 100 minutes \\[2ex]
% \multicolumn{3}{c}{Day 2} \\ \midrule
% Morning   & Invited plenary talk           & 60 minutes \\
%           & Invited talk                   & 30 minutes \\
%           &                                & \grey{\emph{lunch}}  \\
% Morning   & Two invited talks              & 60 minutes \\
%           &                                & \grey{\emph{lunch}}  \\
% Afternoon & Invited plenary talk           & 60 minutes \\
%           & Invited talk                   & 30 minutes \\
%           &                                & \grey{\emph{coffee}} \\
% Afternoon & Four contributed talks         & 100 minutes \\
% \end{tabular}
%   \caption{A possible workshop format}
%   \label{tab:layout}
% \end{table}

\begin{table}[ht]
  \centering
\begin{tabular}{lll}
\multicolumn{3}{c}{Day 1} \\ \midrule
Morning   & Session in honor of Dana Scott & 90 minutes \\
          &                                & \grey{\emph{coffee}}\\
          & Two short talks                & 60 minutes \\
          &                                & \grey{\emph{lunch}} \\
Afternoon & Invited plenary talk           & 60 minutes \\
          & Short talk                     & 30 minutes \\
          &                                & \grey{\emph{coffee}} \\
          & Short talks                    & up to 120 minutes \\[2ex]
\multicolumn{3}{c}{Day 2} \\ \midrule
Morning   & Invited plenary talk           & 60 minutes \\
          & Short talk                     & 30 minutes \\
          &                                & \grey{\emph{lunch}}  \\
Morning   & Two short talks                & 60 minutes \\
          &                                & \grey{\emph{lunch}}  \\
Afternoon & Invited plenary talk           & 60 minutes \\
          & Short talk                     & 30 minutes \\
          &                                & \grey{\emph{coffee}} \\
Afternoon & Short talks                    & up to 120 minutes \\
\end{tabular}
  \caption{A possible workshop format}
  \label{tab:layout}
\end{table}

\subsection{Procedures for selecting papers and participants}

The program committee will select plenary speakers, invited speakers, as well as
contributed papers based on submitted two-page abstracts. The selection of plenary
speakers will reflect the breadth and historic development of domain theory, while the
short invited and contributed talks will be devoted to current topics in domain theory.

% Many other people are also qualified to represent the depth of domain theory, including:
% Abramsky, Berger, Erhard, Edalat, Hyland, Longley, Normann, Simpson.

\subsection{Plans for dissemination}

We plan to produce post-proceedings in a scientifc journal to be determined by the
organizing committee. This is common practice for the Domains Workshop series, which published post-proceedings in journals such as \emph{Mathematical Structure in Computer Science} and \emph{Electronic Notes in Theoretical Computer science}. Other possible journals are \emph{Logical Methods in Computer Science} and \emph{Leibniz International Proceedings in Informatics}.

\subsection{Duration}

Give the breadth and depth of domain theory, as well as Dana Scott's work, and the many
connections of domain theory to logic and computer science, we request \textbf{2 days} for
this workshop. Our preference is for the workshop to take place after LICS and during~FLoC.

\end{document}
